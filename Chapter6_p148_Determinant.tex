\documentclass{ltjsarticle}
\usepackage{amsmath, amssymb}

\begin{document}

\title{第6章 連立方程式が解けるかどうかの判定~行列式~(p.148)}
\maketitle %タイトル表示コマンド

\section*{慣例的な「規則」のLaTeXコード}

\subsection*{1. 余因子展開の形によるグルーピング}

この並べ方は、第1行の成分 $F_{1j}$ でグルーピングする、\textbf{余因子展開}(ラプラス展開)の形式です。行列式をより低い次数の行列式に帰着させる考え方に基づいています。 \\
この方法では、$(\sigma(1)$ の値が 1, 2, 3 の順になるように項を並べます。

$$
\det F = \mathbf{F_{11}}(F_{22}F_{33} - F_{23}F_{32}) - \mathbf{F_{12}}(F_{21}F_{33} - F_{23}F_{31}) + \mathbf{F_{13}}(F_{21}F_{32} - F_{22}F_{31})
$$

% または、余因子展開の行列形式

$$
\det F = F_{11}\det\begin{pmatrix} F_{22} & F_{23} \\ F_{32} & F_{33} \end{pmatrix} - F_{12}\det\begin{pmatrix} F_{21} & F_{23} \\ F_{31} & F_{33} \end{pmatrix} + F_{13}\det\begin{pmatrix} F_{21} & F_{22} \\ F_{31} & F_{32} \end{pmatrix}
$$

\subsection*{2. 辞書式順序によるリストアップ}

置換 $\sigma$ の値 $(\sigma(1)\sigma(2)\sigma(3))$ を\textbf{辞書式順序}($123, 132, 213, \dots$)で並べる形式です。

\begin{center}
\textbf{辞書式順序による置換と項のリスト} \\ % 🌟 キャプションの代わりに太字の見出しを使用
\vspace{2mm} % 🌟 表との間に少しスペースを空ける
\begin{tabular}{|c|c|c|c|}
\hline
$\sigma$ (記法) & $\text{sgn}(\sigma)$ & 行列式の項 & $\sigma$の添字 \\ % \sigmaの添字は元の表にはないが、分かりやすさのために便宜的に追加
\hline
$123$ & $+1$ & $+F_{11}F_{22}F_{33}$ & $\sigma_1$ \\
\hline
$132$ & $-1$ & $-F_{11}F_{23}F_{32}$ & $\sigma_4$ \\
\hline
$213$ & $-1$ & $-F_{12}F_{21}F_{33}$ & $\sigma_2$ \\
\hline
$231$ & $+1$ & $+F_{12}F_{23}F_{31}$ & $\sigma_5$ \\
\hline
$312$ & $+1$ & $+F_{13}F_{21}F_{32}$ & $\sigma_6$ \\
\hline
$321$ & $-1$ & $-F_{13}F_{22}F_{31}$ & $\sigma_3$ \\
\hline
\end{tabular}
\end{center}

$$
\begin{array}{rcl}
\det F &=& \underbrace{F_{11}F_{22}F_{33}}_{\sigma: 123} - \underbrace{F_{11}F_{23}F_{32}}_{\sigma: 132} \\
&& - \underbrace{F_{12}F_{21}F_{33}}_{\sigma: 213} + \underbrace{F_{12}F_{23}F_{31}}_{\sigma: 231} \\
&& + \underbrace{F_{13}F_{21}F_{32}}_{\sigma: 312} - \underbrace{F_{13}F_{22}F_{31}}_{\sigma: 321}
\end{array}
$$

\end{document}