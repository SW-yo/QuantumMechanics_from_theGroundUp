\documentclass{ltjsarticle}
\usepackage{amsmath}
\usepackage{amssymb}
\usepackage{braket} % ケット・ブラ記法のために必要

\begin{document}

\section*{7.8 エルミート演算子の交換と同時固有状態の解説}

この章は、量子力学における「\textbf{同時に測定可能な物理量}」という非常に重要な概念を扱っています。その鍵となるのは、2つの演算子が「\textbf{交換するかどうか}」です。

\subsection*{1. 核心原理:交換条件}

2つのエルミート演算子 $\hat{F}$ と $\hat{G}$ が\textbf{交換する}とは、それらの\textbf{交換子}がゼロになることです。

\[
[\hat{F}, \hat{G}] = \hat{F}\hat{G} - \hat{G}\hat{F} = \hat{0}
\]

\begin{itemize}
    \item 物理的な意味:$\hat{F}$ と $\hat{G}$ に対応する物理量 $F$ と $G$ の測定順序を入れ替えても、系の状態に影響を与えないことを意味します。
    \item 測定可能性:この条件が成立するとき、物理量 $F$ と $G$ は\textbf{同時に正確に測定可能}である、すなわち、\textbf{同時固有状態}が存在します。
\end{itemize}

\subsection*{2. 同時固有状態(同時対角化)}

$\hat{F}$ と $\hat{G}$ が交換する場合、それらは共通の\textbf{固有ベクトル}を持つことができます。これを\textbf{同時固有ベクトル}と呼びます。

\[
\hat{F} \ket{\lambda, \mu} = \lambda \ket{\lambda, \mu} \quad \text{かつ} \quad \hat{G} \ket{\lambda, \mu} = \mu \ket{\lambda, \mu}
\]

\begin{itemize}
    \item 状態 $\ket{\lambda, \mu}$ は、$\hat{F}$ の固有値 $\lambda$ と $\hat{G}$ の固有値 $\mu$ を\textbf{両方}確定的に持つ状態です。
    \item 数学的意義:これは、$\hat{F}$ と $\hat{G}$ を表現する行列が、\textbf{共通の基底}(同時固有ベクトル)によって\textbf{同時に}対角化できることを意味します。
\end{itemize}

\subsection*{3. 非交換(非同時測定)との対比}

もし $\hat{F}$ と $\hat{G}$ が交換しない($[ \hat{F}, \hat{G} ] \ne \hat{0}$)場合、同時固有ベクトルは存在せず、\textbf{不確定性原理}が適用されます。

\begin{itemize}
    \item 例:位置演算子 $\hat{x}$ と運動量演算子 $\hat{p}$ の交換子は $i\hbar$ に比例します。
    \[
    [\hat{x}, \hat{p}] = i\hbar
    \]
    \item 結果:位置と運動量は同時に正確に測定できず、一方を正確に測定しようとすると、もう一方の不確かさが必ず増大します。
\end{itemize}

\end{document}