\documentclass{ltjsarticle}
\usepackage{amsmath}
\usepackage{amssymb}
\usepackage{amsthm}

\begin{document}

ネイピア数 $e$ を底とする指数関数 $f(x) = e^x$ の導関数に関する、非常に重要な性質を示しています。

$$\left(e^x\right)' = e^x$$

\section*{式の意味}

この式は、\textbf{「指数関数 $e^x$ を $x$ で微分しても、その結果は元の関数 $e^x$ のままである」}ということを意味します。

\begin{itemize}
    \item $\left(e^x\right)'$: これは関数 $e^x$ の導関数、つまり $e^x$ を $x$ で微分することを意味します。
    \item $e^x$: ネイピア数 $e$(約 $2.71828$)を底とする指数関数です。
\end{itemize}

したがって、\textbf{指数関数 $e^x$ は、微分しても形が変わらない}という、微分演算において不変であるというユニークな特性を持っています。

\section*{e の定義との関連}

一般に、底が $a$ の指数関数 $f(x) = a^x$ の導関数は
$$(a^x)' = (\ln a) a^x$$
となります。ここで、$\ln a$ は $a$ の\textbf{自然対数}です。

もし導関数が元の関数と等しくなる、つまり $\left(a^x\right)' = a^x$ となるためには、
$$\ln a = 1$$
でなければなりません。この条件を満たす底 $a$ が、まさに\textbf{ネイピア数 $e$} です。

$$\ln e = 1$$

この性質は、ネイピア数 $e$ が持つ最も基本的かつ重要な数学的性質を表現しています。

\end{document}