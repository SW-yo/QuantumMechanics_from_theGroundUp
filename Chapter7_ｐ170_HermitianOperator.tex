\documentclass{ltjsarticle}
\usepackage{amsmath}
\usepackage{amssymb}
\usepackage{amsthm}

\begin{document}

行列 $\hat{F}$ がエルミート演算子(すなわち $\hat{F} = \hat{F}^{\dagger}$)である場合、任意の状態 $|\lambda\rangle$ に対して、次の等式が成り立つことを示します。 \\
$$
\lambda^{*} = \left( \langle\lambda|\hat{F}|\lambda\rangle \right)^{*} = \langle\lambda|\hat{F}^{\dagger}|\lambda\rangle = \langle\lambda|\hat{F}|\lambda\rangle = \lambda
$$
ここで、3番目の式はエルミート共役の定義から導かれ、4番目の式は $\hat{F}$ がエルミート演算子であることを利用しています。 \\


\section*{1. エルミート共役演算子の定義}
一般に、演算子 $\hat{A}$ のエルミート共役 $\hat{A}^{\dagger}$ は、
任意の状態 $|\psi\rangle$ と $|\phi\rangle$ に対して、次の関係を満たす。 \\
footnote{この定義は、量子力学における内積空間の性質に基づいています。}

$$
\langle\phi|\hat{A}|\psi\rangle = \left( \langle\psi|\hat{A}^{\dagger}|\phi\rangle \right)^{*}
$$


\section*{2. 定義式の変形}
上記の定義式の両辺の複素共役を取る。 \\
$$
\left( \langle\phi|\hat{A}|\psi\rangle \right)^{*} = \left( \left( \langle\psi|\hat{A}^{\dagger}|\phi\rangle \right)^{*} \right)^{*}
$$
複素共役の複素共役は元に戻るので、右辺は次のようになる。 \\
$$
\left( \langle\phi|\hat{A}|\psi\rangle \right)^{*} = \langle\psi|\hat{A}^{\dagger}|\phi\rangle
$$


\section*{3. 最初の式への適用}
 $|\phi\rangle \to |\lambda\rangle$, $|\psi\rangle \to |\lambda\rangle$, $\hat{A} \to \hat{F}$ と置き換える。 \\
$$
\left( \langle\lambda|\hat{F}|\lambda\rangle \right)^{*} = \langle\lambda|\hat{F}^{\dagger}|\lambda\rangle
$$


\section*{4. エルミート演算子との関係}
$\hat{F}$ がエルミート演算子($\hat{F} = \hat{F}^{\dagger}$)の場合、3番目の式は4番目の式に進む。 \\
$$
\underbrace{\langle\lambda|\hat{F}^{\dagger}|\lambda\rangle}_{\text{3番目の式}} \quad \xrightarrow{\text{エルミート演算子の性質}} \quad \underbrace{\langle\lambda|\hat{F}|\lambda\rangle}_{\text{4番目の式}}
$$


\section*{5. まとめ}
$$
\left( \langle\lambda|\hat{F}|\lambda\rangle \right)^{*} = \langle\lambda|\hat{F}^{\dagger}|\lambda\rangle \quad \text{(常に成立)}
$$
$$
\langle\lambda|\hat{F}^{\dagger}|\lambda\rangle = \langle\lambda|\hat{F}|\lambda\rangle \quad (\hat{F} \text{ がエルミート演算子の場合に成立})
$$

\end{document}